\documentclass{article}
\usepackage[margin=0.75in]{geometry}

\title{Trading Strategies Guide}
\author{Varun Varanasi}

\begin{document}
\maketitle

\section{General Trading Strategies}
High level overview of contributing factors to day trading decisions

\subsection{Identifying Stocks}
\begin{itemize}
    \item Liquidity: Liquid stocks allow you to enter and exit the market easily; liquidity also assists in trading under conditions of low slippage (difference between expected and actual price) and tight spreads
    \item Volatility: A metric for variability of the stock
    \item Trading Volume: Trading volume gives you insight into demand for the stock
\end{itemize}

\subsection{Entry Points}
\begin{itemize}
    \item 
\end{itemize}

\subsection{Exit Points}
\begin{itemize}
    \item Scalping: Exiting your position as soon as it is profitable
    \item Fading: Shorting stocks after rapid movements upwards
    \item Daily Pivots: Using volatility predictions to buy low and sell high
    \item Momentum: Riding trends until they appear to reverse
\end{itemize}

Long term investing tends to be determined by fundamental analysis (capitalize on differences between an asset's price and intrinsic value)


\subsection{Evaluating Trading Strategies}
\subsubsection*{Profit Net Loss}
Evaluate trading strategies based on the profits and net lose incurred over a pre-determined time period. While the most simple evaluation of a trading strategy, this method overlooks risk.

\subsubsection*{Sharpe Ratio}

The Sharpe Ratio is used to evaluate an investment in terms of returns and risk. 

$$
Sharpe \ Ratio = \frac{R_p - R_f}{\sigma_p}
$$
where $R_p$ is the return of the portolio, $R_f$ is rate of risk-free return, and $\sigma_p$ is the standard deviation of the portfolio's excess return.



\section{Algorithmic Trading Strategies}

\end{document}