\documentclass{article}
\usepackage[margin=0.75in]{geometry}
\usepackage{booktabs}
\usepackage{caption}

\title{Trading Strategies Summary}
\author{Varun Varanasi}

\begin{document}
\maketitle

The goal of this document is to outline different trading strategies implemented with the Metaculus data on the VIX, SPY, and their derivatives. Each strategy only trades 1 stock per action. 

\section*{Metaculus Momentum}
This simple strategy involves opening a position when the Metaculus series increases and closing the position when it decreases. This strategy only allows for one open position at a time.

\begin{table}[h]
\centering

\begin{tabular}{l||ccc}
    \toprule
     & \textbf{VXX} & \textbf{SPY} & \\
    \midrule
    PnL & \$38.10 & \$86.85 \\
    Sharpe Ratio & -2.06 & -4.89\\
    \bottomrule
\end{tabular}
\caption{Summary of results from 1/1/2020 to 4/1/2022}
\end{table}

\section*{Metaculus Moving Average}
This strategy involves a slow and a fast moving average indicator on the Metaculus series. For the SPY, when the fast moving average passes the slow moving average (indicating populace uncertainity), we short as we expect volatility to increase. We then close our positions when the slow moving average passes the fast one.
Similarly, for the VXX, we trade using the opposite strategy. The fast moving average passing the slow one is interpretted as a signal to buy while the slow passing the fast is a signal to close the position. In both cases, if the position is open for a pre-derminted hold period, it is immediately closed. We ran optimization on the slow, fast, and hold-periods to find the following results:

\begin{table}[h]
    \centering
    
    \begin{tabular}{l||ccc}
        \toprule
         & \textbf{VXX} & \textbf{SPY} & \\
        \midrule
        Slow Moving Average & 67  & 50\\
        Fast Moving Average & 8  & 16\\
        Hold Period & 7  & 17\\
        PnL & \$80.10 & \$100.28 \\
        Sharpe Ratio & -3.37 & -1.33\\
        \bottomrule
    \end{tabular}
    \caption{Summary of results from 1/1/2020 to 4/1/2022}
    \end{table}

\section*{Metaculus Bollinger Bands}


\end{document}