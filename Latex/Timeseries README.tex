\documentclass{article}
\usepackage[margin=0.75in]{geometry}
\usepackage{hyperref}
\usepackage{amsfonts}

\title{Timeseries README} 
\author{Varun Varanasi}

\begin{document}

\maketitle

The goal of this document is to summarize the datatypes and functionality of the TimeSeriesPlot.py script. 
This script assumes that it is in the same working directory as the Data folder which includes specially named files required for the functions to work.
Data in the data folder is updated according to the the DownloadCSV.py script which is run hourly by a cronjob. 

\section*{Functions}

\subsection*{ts\_plot}
This function creates a timeseries plot given a finanical index, a start date, and variables to plot. \\
\textbf{Inputs:}
\begin{itemize}
    \item \textbf{data\_name}: Name of finanicial index you would like to plot (see Available Datasets) 
    \item \textbf{start\_date}: Start date of the timeseries (syntax note:  YYYY-MM-DD format)
    \item \textbf{variables}: Select which variables from the dataset you would like to plot (syntax note: put all variables in a list)
    \item \textbf{fill}: Select 1 to create a fill plot (timeseries value duplicated over x-axis and filled in to better visualize change)
\end{itemize}
\textbf{Outputs:} A series of python plots detailing the timeseries at the specified conditions and variables
\\

\subsection*{ts\_decomp}
This function decomposes a given financial index into its trend, seasonal, and noise components by both multiplicative and additive methods. \\
\textbf{Inputs:}
\begin{itemize}
    \item \textbf{data\_name}: Name of finanicial index you would like to plot (see Available Datasets) 
    \item \textbf{start\_date}: Start date of the timeseries (syntax note:  YYYY-MM-DD format)
    \item \textbf{variables}: Select which variables from the dataset you would like to plot (syntax note: put all variables in a list)
    \item \textbf{period}: Determines what the period of the timeseries analysis is
\end{itemize}
\textbf{Outputs:} A series of python plots detailing the timeseries decomposition at the specified conditions and variables.
\\

\subsection*{ADH\_test}
The ADH Test is a test of stationarity of a timeseries. This function runs the ADH test on given financial index, start date, and variables. 
Generally, the rejection criteria is at a p-value less than 0.05\\
\begin{itemize}
    \item \textbf{data\_name}: Name of finanicial index you would like to plot (see Available Datasets) 
    \item \textbf{start\_date}: Start date of the timeseries (syntax note:  YYYY-MM-DD format)
    \item \textbf{variables}: Select which variables from the dataset you would like to plot (syntax note: put all variables in a list)
\end{itemize}
\textbf{Outputs:} Variable name, ADF statistic, associated p-value and critical values. 

\subsection*{seasonality\_test}
The seasonality\_test function creates a plot detailing the seasonality of the timeseries given. Expected output will highlight peaks at regular intervals corresponding with the seasonality of the data. 
\begin{itemize}
    \item \textbf{data\_name}: Name of finanicial index you would like to plot (see Available Datasets) 
    \item \textbf{start\_date}: Start date of the timeseries (syntax note:  YYYY-MM-DD format)
    \item \textbf{variables}: Select which variables from the dataset you would like to plot (syntax note: put all variables in a list)
\end{itemize}
\textbf{Outputs:} Python plot showing an autocorrelation plot for the given timeseries. 


\subsection*{interpolate\_data}
This function adds granularity at a specified interval to the provided timeseries.  

\begin{itemize}
    \item \textbf{data\_name}: Name of finanicial index you would like to plot (see Available Datasets) 
    \item \textbf{start\_date}: Start date of the timeseries (syntax note:  YYYY-MM-DD format)
    \item \textbf{variables}: Select which variables from the dataset you would like to plot (syntax note: put all variables in a list)
    \item \textbf{intervals}: Select which interval of granularity you would like to interpolate (30S: 30 second intervals, 1H: 1 hour intervals, etc.)
    \item \textbf{methods}: Select between a "Linear" or a "Seasonal" interpolation (seasonal interpolation is a linear interpolation with random noise corresponding to seasonal variation of the timeseries)
    \item \textbf{period}: Select the period for the timeseries seasonality analysis
    \item \textbf{csv}: Select 1 to save a .csv file of your interpolated data in the "Data" folder
\end{itemize}
\textbf{Outputs:} Saves .csv of interpolated data in "Data" folder and produces a plot of the interpolated timeseries


\section*{Available Datasets}

\subsection*{VIX}
\begin{itemize}
    \item VIX is a measure of S\&P 500 volatility 30 days from the date in question
    \item The VIX dataset updated daily and labeled as \textbf{VIX\_History\_DATE\_OF\_UPDATE.csv} in the Data folder
    \item The dataset contains \textbf{date, OPEN, HIGH, LOW, CLOSE} for each day
\end{itemize}

\subsection*{VIX9D}
\begin{itemize}
    \item VIX9D is a measure of S\&P 500 volatility 9 days from the date in question
    \item The VIX9D dataset updated daily and labeled as \textbf{VIX9D\_History\_DATE\_OF\_UPDATE.csv} in the Data folder
    \item The dataset contains \textbf{date, OPEN, HIGH, LOW, CLOSE} for each day
\end{itemize}

\subsection*{VVIX}
\begin{itemize}
    \item VVIX is a VIX index on the VIX itself
    \item The VVIX dataset is updated daily and labeled as \textbf{VVIX\_History\_DATE\_OF\_UPDATE.csv} in the Data folder
    \item The data set contains \textbf{date, VVIX}
\end{itemize}

\subsection*{VXX}
\begin{itemize}
    \item VXX is a short-term future derivative of the VIX
    \item The VXX dataset is updated daily and labeled as \textbf{VXX\_History\_DATE\_OF\_UPDATE.csv} in the Data folder
    \item The data set contains \textbf{Date, Open, High, Low, Close, Adj Close, Volume}
\end{itemize}

\subsection*{VIX Futures}
As of now this data set has not been implemented into the TimeSeriesPlot.py script. 




\end{document}