\documentclass{article}
\usepackage[margin=0.75in]{geometry}
\usepackage{hyperref}

\title{VIX Summary}
\author{Varun Varanasi}
\date{February 26th, 2022}

\begin{document}
\maketitle


\section*{Key Takeaways}
\begin{itemize}
    \item VIX index is a measure of market volatility based on S\&P 500 index option bid/ask prices
    \item VIX cannot be traded, but its derivatives can (i.e VXX)
    \item VIX index calculates volatility of the S\&P 500 30 days out, but there exist a family of VIX indicies which calculate volatility for different time periods
    \item VIX is a function of near term options, next-term options, and risk free treasury bill interest rates
    \item VIX index is calculated and updated at 15 second intervals by a deterministic formula and selection algorithim
    \item VIX tends to be inversely correlated with S\&P 500
    \item There is free historic data of VIX, VIX derivatives, and VIX family indicies that is updated daily; however, that data is generally limited to EOD levels.
    \item CBOE sells VIX data at 15 second intervals by the month for 10 dollars per dataset 
\end{itemize}

\section*{General Notes}
\begin{itemize}
    \item Operated by the Chicago Board Options Exchange, the VIX index is an up-to-the-minute measure of market volatility based on the midpoints of S\&P 500 index option bid/ask quotes
    \item The current VIX index measures the expected annualized change in the S\&P 500 index in the next 30 days as calculated by options pricings
    \item Essentially, the VIX is a measure of the markets prediction of S\&P 500 volatility 30 days in the future
    \item \textbf{VIX cannot be traded}; instead, it is traded through derivative contracts like futures, ETFs, and options
    \item Related CBOE Products:
    \begin{itemize}
        \item VIX9DSM (Short Term Volatility Index)
        \item VIX3MSM (3 Month Volatility Index)
        \item VIX6MSM (6 Month Volatility Index)
        \item VIX1YSM (1 Year Volatility Index)
        \item VXNSM (NASDAQ-100 Volatility Index)
        \item VXDSM (DOW Volatility Index)
        \item RVXSM (Russell 2000 Volatility Index)
        \item VVIX (VIX on VIX)
    \end{itemize}
    \item VIX is a function of S\&P 500 index options for near term options (less than 23 days till expiry), next-term options (less than 37 days till expiry), and risk free U.S. Treasury Bill interest rates
    \item The VIX is the volatility of variance swap
    \subsection*{Quick Interlude into Variance Swaps}
    \begin{itemize}
    \item Variance Swap is a derivative contract dependent on the changes in the underlying asset's price (volatility)
    \item Essentially, one party agrees to pay the implied variance (strike) to the other party who will then pay back the actual variance at the end of the contract
    \item Profits are realized when the actual variance is greater than the implied variance (calculated from options contracts)
    \item Variance swaps function similarly to options contracts; however, since option contracts are dependent on directionality, time, expiration, and implied volatility, variance swap contracts tend to be less risky
    \item Actual variance is calculated from daily log returns (natural log of the difference between given day and previous day price)
    \end{itemize}
    \item The VIX index is calculated as the square root of a risk-neutral expectation of the S\&P 500 variance over 30 days and is quoted as an annualized standard deviation
    \item VIX is calculated by a weighted portfolio of out-of-the-money European-style options of the S\&P 500
    \begin{itemize}
        \item Unlike their American counterparts, European options can only be exercised on the expiration date rather than at anytime before it
        \item Most index options are European-style options
        \item Black-Scholes Equation models European-style options
    \end{itemize} 
    \item VIX Formula:
    \begin{equation}
        VIX= 100 *\sqrt{\frac{2e^{r\tau}}{\tau}\left(\int_0^F \frac{P(K)}{K^2}dK + \int_0^\infty \frac{C(K)}{K^2}dK\right)}
    \end{equation}
    \begin{itemize}
        \item $\tau$ is the average days in a month (30)
        \item r is the risk-free rate (Calculated by subtracting inflation rate from the yield of a US treasury bond matching investment duration)
        \item F is the 30 day forward price on the S\&P 500
        \item P(K) is the price for a put of with strike price K 30 days from expiration
        \item C(K) is similarly the price for a call with a strike price of K 30 days from expiry
        \item More details on how the individual variables are calculated can be found in the whitepaper: \url{https://cdn.cboe.com/api/global/us_indices/governance/VIX_Methodology.pdf}
    \end{itemize}
    \item VIX is only calculated based on options expiring on Fridays
    \item Intraday VIX index values are calculated based on the bid/ask quotes of SPX options at 15 second intervals
    \item In order to filter extraneous movement in option pricing, the VIX index is filtered before being updated
    \begin{itemize}
        \item An initial baseline VIX index value is calculated
        \item The baseline is updated if the VIX index moves by more than 0.49 in each direction within 2 minutes of the original baseline calculation
        \item If there is no change after the 2 minute period, the VIX index is updated to the first value calculated after the 2 minute interval
    \end{itemize} 
    \item Generally, the VIX index is inversely correlated with S\&P 500 Index
    \item The VIX Index tends to be used for risk management or hedging and usually tends towards its historical average
    \item VIX Futures contracts 
    \begin{itemize}
        \item VXX: Long position in 1st and 2nd month VIX futures that roll daily
        \begin{itemize}
            \item Negative Roll Yield (long-term holders face penalties on returns)
            \item Stronger in short-term than long-term
        \end{itemize}
        \item VXZ: Long position in fourth-, fifth-, sixth-, and seventh- month VIX futures
        \begin{itemize}
            \item Negative Roll Yield (long-term holders face penalties on returns)
            \item Stronger in short-term than long-term
        \end{itemize}
    \end{itemize}
    \item Criticism against the efficacy of the VIX index
    \begin{itemize}
        \item Some critics say that the VIX is just a meausre of current pricing of index options
        \item Similar to simpiler volatility metrics (past volatility)
        \item Lacks predictive power and simply just the inverse of pricing
    \end{itemize}

\end{itemize}

    
\section*{Datasets}
\begin{itemize}
    \item VIX Index Values by 15 seconds: \url{https://datashop.cboe.com/volatility-index-spot-values}
    \begin{itemize}
        \item Costs 10 dollars per month
    \end{itemize}
    \item VIX Future Trades: \url{https://datashop.cboe.com/cfe-vix-volatility-index-futures-trades-quotes}
    \begin{itemize}
        \item Costs 50 dollars per month
    \end{itemize}
    \item VIX Index Values: \url{https://www.cboe.com/tradable_products/vix/vix_historical_data/}
    \begin{itemize}
        \item Free and updated daily
        \item Only Open, High, Low, and Close data
        \item Also available through Yahoo Finance: \url{https://finance.yahoo.com/quote/%5EVIX/history?p=%5EVIX}
    \end{itemize}
    \item VIX of VIX Index: \url{https://www.cboe.com/tradable_products/vix/vix_historical_data/}
    \begin{itemize}
        \item Free and updated daily
        \item Only EOD
    \end{itemize}
    \item 9-day VIX Index: \url{https://www.cboe.com/tradable_products/vix/vix_historical_data/}
    \begin{itemize}
        \item Free and updated daily
        \item Only Open, High, Low, and Close data
    \end{itemize}
    \item VX Futures: \url{https://www.cboe.com/us/futures/market_statistics/historical_data/}
    \begin{itemize}
        \item Free and contains futures contract data
    \end{itemize}
    \item VIX Options: \url{https://www.cboe.com/us/options/market_statistics/historical_data/}
    \item VXX: \url{https://finance.yahoo.com/quote/VXX/history?p=VXX}
    \begin{itemize}
        \item Daily historical data
    \end{itemize}
    
\end{itemize}




\end{document}